\chapter{Objetivos y metodología de Trabajo}
\label{cap:capitulo2}

\section{Objetivos}
\label{sec:objetivos}

En la herramienta VisualCircuit, antes de realizar este proyecto, ya existían bloques dedicados específicamente a la robótica para algunos sensores
(cámara, odometría e IMU\footnote{\textbf{IMU}: Inertial Measurement Unit}) y para los motores usando ROS
\textit{Noetic}\footnote{\textbf{ROS Noetic}: \url{http://wiki.ros.org/noetic}}, pero al tratarse de una versión obsoleta del middleware,
decidimos que era momento de actualizar a ROS2 \textit{Humble}\footnote{\textbf{ROS Humble}: \url{https://docs.ros.org/en/humble/index.html}},
ya que es la versión estable más moderna en el momento de realización de este Trabajo Fin de Grado.\\

Los objetivos concretos de este Trabajo Fin de Grado son los siguientes:

\begin{itemize}
	\item   Desarrollar bloques para sensores y actuadores usando ROS2 \textit{Humble} y probar su correcto funcionamiento mediante
                        circuitos simples de VisualCircuit.
                    Profundizaremos más en el capítulo \ref{cap:capitulo4}.
	\item   Diseñar y construir aplicaciones complejas que usen estos bloques para comprobar su funcionalidad en situaciones reales:
    \begin{itemize}
        \item   Aplicación sigue-persona: usar reconocimiento visual para seguir a una persona tanto en entorno simulado como real,
                    usando el robot TurtleBot2 (sección \ref{sec:turtlebot2}).
                    Veremos más en el capítulo \ref{cap:capitulo5}.
        \item   Aplicación \textit{Virtual Force Field}: usar el láser y la odometría para navegar por el entorno evitando obstáculos.
                    Examinaremos con mayor detalle en el capítulo \ref{cap:capitulo6}.
    \end{itemize}
\end{itemize}

\newpage

\section{Metodología}
\label{sec:metod}

Este Trabajo Fin de Grado comenzó en abril de 2022 y finalizó en junio de 2023. Durante estos meses se ha seguido el siguiente modelo de trabajo: 

\begin{itemize}
	\item   Reuniones cada dos semanas con el tutor del TFG para analizar los avances, recibir retroalimentación y buscar soluciones en caso
                de bloqueo.
	\item   Uso de un blog\footnote{\textbf{Blog}: \url{https://roboticslaburjc.github.io/2022-tfg-david-tapiador/}} donde se iba actualizando el
                progreso antes de las reuniones, donde se puede comprobar el desarrollo cronológico del trabajo.
	\item   Todo el material usado y desarrollado durante este Trabajo Fin de Grado se ha ido actualizando en un repositorio público de
                GitHub\footnote{\textbf{GitHub del TFG}: \url{https://github.com/RoboticsLabURJC/2022-tfg-david-tapiador}}.
\end{itemize}

\section{Plan de Trabajo}
\label{sec:work_plan}

Como se ha dicho en el anterior punto, el desarrollo de este TFG ha durado algo más de un año. Este periodo se ha dividido en varias etapas: 

\begin{enumerate}
	\item   \textbf{Pruebas con VisualCircuit}: Realizar circuitos para entender el funcionamiento de la plataforma, desarrollando bloques
                propios que modifiquen imágenes o compartan información.
	\item   \textbf{Inicio del TFG}: Configuración del entorno de pruebas (instalación de ROS2 \textit{Humble}, diseño de mundos
                en \textit{Gazebo}...)..
    \item   \textbf{Desarrollo de bloques drivers con ROS2}: Actualización y creación de bloques para sensores y actuadores usando ROS2 \textit{Humble}.
    \item   \textbf{Prueba de los bloques en situaciones reales}: Desarrollo de aplicaciones robóticas avanzadas para probar el correcto
                funcionamiento de los bloques implementados. 
    \item   \textbf{Memoria del Trabajo Fin de Grado}: Redacción de esta memoria. 
\end{enumerate}

