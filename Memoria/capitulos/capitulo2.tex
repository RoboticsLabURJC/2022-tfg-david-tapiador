\chapter{Objetivos y metodología de Trabajo}
\label{cap:capitulo2}

\begin{flushright}
\begin{minipage}[]{10cm}
\emph{Quizás algún fragmento de libro inspirador...}\\
\end{minipage}\\

Autor, \textit{Título}\\
\end{flushright}

\vspace{1cm}

Escribe aquí un párrafo explicando brevemente lo que vas a contar en este capítulo. En este capítulo lo ideal es explicar cuáles han sido los objetivos que te has fijado conseguir con tu trabajo, qué requisitos ha de respetar el resultado final, y cómo lo has llevado a cabo; esto es, cuál ha sido tu plan de trabajo.\\

\section{Descripción del problema}
\label{sec:descripcion}

Cuenta aquí el objetivo u objetivos generales y, a continuación, concrétalos mediante objetivos específicos.

\section{Requisitos}
\label{sec:requisitos}

Describe los requisitos que ha de cumplir tu trabajo.

\section{Metodología}
\label{sec:metodologia}

Qué paradigma de desarrollo software has seguido para alcanzar tus objetivos.

\section{Plan de trabajo}
\label{sec:plantrabajo}






***COSAS GUARDADAS PARA HACER***

-En la revision del cap 4 pone cómo hacer este cap

-Cacho guardado para hacer el capitulo2.

********
En VisualCircuit, antes de realizar este proyecto, ya existían bloques dedicados específicamente a la robótica para algunos sensores
(cámara, odometría e IMU\footnote{\textbf{IMU}: Inertial Measurement Unit}) y para los motores usando ROS
Noetic\footnote{\textbf{ROS Noetic}: \url{http://wiki.ros.org/noetic}}, pero al tratarse de una versión antigua, decidimos que ya era
momento de actualizar a ROS2 Humble\footnote{\textbf{ROS Humble}: \url{https://docs.ros.org/en/humble/index.html}}, ya que era la versión estable
más moderna hasta el momento.
******












