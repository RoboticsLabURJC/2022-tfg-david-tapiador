\chapter{Conclusiones}
\label{cap:capitulo7}

Tras detallar en profundidad las mejoras aportadas a la plataforma VisualCircuit, en este último capítulo de la memoria del
Trabajo Fin de Grado se hará un resumen de las metas logradas.

\section{Conclusiones}

El objetivo principal de este proyecto era actualizar la plataforma VisualCircuit para que incluyera bloques orientados directamente a
la robótica usando ROS2 \textit{Humble}. Este objetivo ha sido alcanzado con éxito, añadiendo a las bibliotecas de VisualCircuit los bloques
correspondientes a los sensores (láser y cámara) y actuadores (motores) tanto reales como simulados.
También se ha desarrollado un bloque \textit{encoders} para obtener la posición del robot en el simulador.\\

Otro de los objetivos era comprobar el correcto funcionamiento de estos bloques mediante aplicaciones simples que lo demostraran. Como
hemos comprobado en el capítulo \ref{cap:capitulo4}. También se buscaba desarrollar aplicaciones complejas que demostraran la eficacia de
los bloques, y mediante la aplicación sigue-personas (\ref{cap:capitulo5}) y \textit{Virtual Force Field} (\ref{cap:capitulo6}) se puede
comprobar que este objetivo también se ha cumplido.\\

Este TFG ha logrado enriquecer la plataforma VisualCircuit permitiendo desarrollar aplicaciones más avanzadas y acercarla cada vez más a
ser una de las plataformas online de referencia para programar robots.

\newpage

\section{Líneas futuras}

Algunas formas en las que se podría continuar con el desarrollo planteado en este Trabajo Fin de Grado podrían ser:

\begin{itemize}
	\item   Desarrollar bloques para nuevos sensores, como podrían ser los \textit{bumpers}, o actuadores como \textit{leds} y sonidos.
	\item   Avanzar hacia el camino de los drones, incluyendo bloques que activen, desactiven y configuren los modos de vuelo y que
                permitan despegar y aterrizar, ya que la programación de drones suele ser más tediosa y ver el comportamiento
                de forma tan visual como permite VisualCircuit ayudaría en el proceso.
	\item   Ampliar VisualCircuit de tal forma que se permita compartir bloques y comportamientos completos entre distintos usuarios
                mediante un foro comunitario, añadiendo más bloques prefabricados a las bibliotecas de bloques.
\end{itemize}
