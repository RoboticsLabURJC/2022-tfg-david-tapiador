\chapter{Herramientas y plataforma de desarrollo}
\label{cap:capitulo3}

El desarrollo de nuevo contenido para la plataforma de VisualCircuit, ha necesitado usar distintas herramientas, como por ejemplo programación, ROS2, gazebo..., por lo que voy a hacer una pequeña descripción de la utilidad de cada una, así como el uso que se le ha dado dentro del proyecto.

\section{Lenguaje de programación}
\label{sec:lenguaje_programación}
% ** LENGUAJES DE PROGRAMACÍON
% ** PYTHON

Python es un lenguaje interpretado de alto nivel. Este lenguaje busca facilitar la legibilidad del código, convirtiéndolo en uno de los más comunes a día de hoy. Es un lenguaje de programación multiparadigma, ya que soporta tanto programación orientada a objetos, como programación imperativa y funcional.\\

\begin{code}[H]
    \begin{lstlisting}[language=python]
    print("Hello World")
    \end{lstlisting}
    \caption[Hola mundo en python]{Hola mundo en python}
    \label{cod:holamundo_python}
\end{code}

Dentro del TFG se usará para la programación dentro de la plataforma VisualCircuit (\ref{sec:visualcircuit}).

\section{ROS2 (Robot Operating System 2)}
\label{sec:ros2}
% ** MIDDLEWARE ROS2

ROS\footnote{\textbf{ROS}: \url{http://wiki.ros.org/es}} o Robot Operating System es un \textit{middleware}\footnote{{\textbf{Middleware}: software que se sitúa entre las aplicaciones y el sistema operativo}} formado por un conjunto de herramientas y librerías de software libre empleadas para el desarrollo de aplicaciones robóticas. Su objetivo es ofrecer una plataforma estándar para todas las ramas de la robótica.\\

ROS se basa en una arquitectura \textit{cliente-servidor} centralizado que, mediante suscriptores y publicadores, permite enviar información, ya sean medidas de sensores, cambios de estado, decisiones usando árboles de decisión, órdenes a los actuadores, etc.\\

Para comunicarse con los servidores (o como se llaman en ROS, "\textit{topics}) se usan nodos. Estos nodos pueden contar con varios publicadores y suscriptores simultáneaos.
Cada \textit{topic} se define con un tipo de mensaje, que será el único que se pueda enviar y recibir a través de él. Estos tipos de mensajes pueden ser mensajes simples como una cadena de caracteres o tipos compuestos con otros tipos, permitiéndonos crear topics adecuados a las necesidades de cada proyecto.\\

\begin{figure} [H]
    \begin{center}
        \includegraphics[width=7cm]{figs/c3/ros_comunicación.png}
    \end{center}
    \caption[Comunicación del nodo Master con los nodos Intermedios y con distintos sensores y actuadores.]{Comunicación del nodo Master con los nodos Intermedios y con distintos sensores y actuadores.. Imagen obtenida de \cite{comunicacion_ros2}}
    \label{fig:ros_master_comunicacion}
\end{figure}

ROS2 lo usaremos para obtener información del robot turtlebot2 (\ref{sec:turtlebot2}), tanto real como simulado, como por ejemplo su posición en el entrono simulado o las últimas medidas de sus sensores, y para comandarle instrucciones (velocidades a sus motores)

\section{Gazebo}
\label{sec:gazebo}
% ** GAZEBO

Gazebo\footnote{\textbf{Gazebo}: \url{https://classic.gazebosim.org/}} es un simulador 3D de código abierto orientado a la robótica que permite fusionar escenarios realistas con robots simulados, ofreciendo un entorno seguro para probar algoritmos. Éste utiliza el motor de físicas ODE\footnote{\textbf{Open Dynamics Engine}: \url{https://www.ode.org/}}, aunque se puede configurar con otros motores, como Bullet\footnote{\textbf{Bullet}: \url{https://pybullet.org/wordpress/}} o DART\footnote{\textbf{DART}: \url{https://dartsim.github.io/}}.\\

Al estar orientado a la robótica, permite integrar fácilmente modelos de robots reales con sensores (incluso simulando sus ruidos) y enviar a través de los distintos topics de ROS o ROS2 (\ref{sec:ros2}) alguna información directa del simulador, como la posición, medidas de los sensores simulados o incluso información de objetos no programables (del entorno).

\begin{figure} [H]
    \begin{center}
        \includegraphics[width=7cm]{figs/c3/gazebo_sim.png}
    \end{center}
    \caption[Simulador Gazebo.]{Ejemplo de ejecución en gazebo.}
    \label{fig:gazebo_example}
\end{figure}

A lo largo de todo el proyecto, usaremos gazebo para simular el turtlebot2 (\ref{sec:turtlebot2}) al igual que los distintos entornos que iremos usando para probar los programas.
 
\section{Turtlebot2}
\label{sec:turtlebot2}
% ** TURTLEBOT2

La URJC de Fuenlabrada, en sus laboratorios de robótica, cuenta con varios robots Turtlebot2\footnote{\textbf{Turtlebot2}: \url{https://www.turtlebot.com/turtlebot2/}} a disposición de los alumnos del grado. Estos son perfectos para la enseñanza e investigación en robótica, por su sencilla introducción a temas como ROS o el uso de sensores. Los Turtlebot2 están formados por dos partes principales: una base Kobuki y una estructura superior.


\subsection{Base Kobuki}
\label{subsec:turtlebot2_base}
% ** TURTLEBOT2 -> KOBUKI

La base del Turtlebot2 se llama \textit{Kobuki}. En apariencia, es similar a un robot de limpieza como podrían ser los Roomba. En cuanto al hardware, lleva integrados tres bumpers (sensores de contacto), odometría, sensor de caída y varios giroscopios. Tiene una velocidad lineal máxima de 0.7 m/s y angular de 180 grados/s. Su batería le permite una autonomía de entre 3 y 7 horas. Cuenta con varios puertos, entre ellos un USB para poder conectar nuestro portatil y ejecutar los distintos algoritmos.\\

Algunos de los paquetes de ROS2 que instalaremos para poder usarlo son los drivers del kobuki para ROS2-Humble\footnote{\textbf{Drivers kobuki ROS2-Humble}: \url{https://github.com/IntelligentRoboticsLabs/Robots/tree/humble/kobuki}} de IntelligentRoboticsLabs, compañeros de la URJC. Siguiendo las instrucciones de instalación que se encuentran en dicho repositorio de github, accedemos a varios paquetes básicos para el uso de kobuki, como \textit{kobuki\_ros} o \textit{kobuki\_node}, entre otros.

\begin{figure} [H]
    \begin{center}
        \includegraphics[width=7cm]{figs/c3/kobuki_base.jpg}
    \end{center}
    \caption[Kobuki base]{Base kobuki. Imagen obtenida de \cite{kobuki_base}}
    \label{fig:kobuki_base}
\end{figure}

\subsection{Cuerpo Turtlebot2}
\label{subsec:turtlebot2_body}
% ** TURTLEBOT2 -> CUERPO
El cuerpo del turtlebot2 (también conocido como \textit{TurtleBot Structure}) está formado por una serie de plataformas y tubos que se atornillan a la base kobuki y permiten fijar nuevos sensores, como podrían ser una cámara o un láser, o actuadores como brazos robóticos. También ofrece un sitio cómodo para poder colocar el portátil encima del robot y así poder conectarlo a la base mediante USB. 


\begin{figure} [H]
    \begin{center}
        \includegraphics[width=5cm]{figs/c3/turtlebot2_body.jpg}
    \end{center}
    \caption[Turtlebot2]{Turtlebot2. Imagen obtenida de \cite{turtlebot_2_structure}}
    \label{fig:turtlebot_2_structure}
\end{figure}

\subsection{Turtlebot2 simulado}
\label{subsec:turtlebot2_sim}
% ** TURTLEBOT2 -> SIMULADOR

Para algunas partes del proyecto, como el desarrollo de drivers para ROS2 (\ref{cap:capitulo4}) o el VFF usando máquinas de estados (\ref{cap:capitulo6}), hemos usado el simulador para probar y desarrollar los algoritmos. Para esto, he tenido que usar un modelo del turtlebot2 que cuenta con los mismos sensores (cámara, RPLIDAR, bumper, etc) que el real, así como los mismos topics.\\

Para integrar el modelo del robot en el simulador, necesitamos su representación en URDF\footnote{\textbf{URDF}: Unified Robot Description Format}, una forma estandarizada de crear los modelos de los robots inluyendo sus sensores y actuadores, partes móviles etc.

\begin{figure} [H]
    \begin{center}
        \includegraphics[width=14cm]{figs/c3/turtlebot2_sim.png}
    \end{center}
    \caption[Turtlebot2 simulado]{Turtlebot2 en gazebo.}
    \label{fig:turtlebot_2_sim}
\end{figure}


\newpage

\section{RVIZ2}
\label{sec:rviz2}
% ** RVIZ2

RVIZ2 es una herramienta de visualización 3D para robots, el ambiente y las medidas de los sensores de éstos.

\begin{figure} [H]
    \begin{center}
        \includegraphics[width=13cm]{figs/c3/RVIZ2.png}
        \includegraphics[width=13cm]{figs/c3/Gazebo_RVIZ.png}
    \end{center}
    \caption[RVIZ2 Vs mundo gazebo]{Ejemplo de RVIZ2 frente al mundo gazebo real.}
    \label{fig:rviz2_example}
\end{figure}

RVIZ2 se usará bastante durante el proyecto tanto para depurar como para observar las medidas de los distintos sensores a tiempo real.

\newpage

\section{Sensores}
\label{sec:sensores}
% ** SENSORES

Como hemos mencionado anteriormente, un robot se compone, a grandes rasgos, de sensores y actuadores. Para este proyecto se han usado varios de ellos, por lo que aquí hay una pequeña introducción a cada uno.

\subsection{Cámara ASUS Xtion Pro}
\label{subsec:asus_xtion}
% ** ASUS XTION
La cámara \textit{ASUS Xtion} es una cámara RGB-D\footnote{\textbf{RGB-D}: RedGreenBlue-Depth, hace referencia las cámaras que captan la imagen y las distancias de cada pixel.}, que ofrece tanto imagen como una nube de puntos con la distancia medida para cada pixel de la imagen.
Esta cámara ofrece una imágen de 720p, con una frecuencia de 60fps. En la parte de profundidad, es capaz de captar desde 0.8m hasya 3.5 con un ángulo efectivo de 70º. Se conecta mediante USB directamente al ordenador.\\
En el proyecto, como debemos usarla con ROS2, usaremos el paquete creado por un usuario de internet\footnote{\textbf{Drivers ASUS-Xtion ROS2}: \url{https://github.com/mgonzs13/ros2_asus_xtion}}.\\
\begin{figure} [H]
    \begin{center}
        \includegraphics[width=5cm]{figs/c3/asus_xtion.jpg}
    \end{center}
    \caption[Cámara ASUS-XTION]{Cámara ASUS-XTION. Imagen obtenida de \cite{asus_xtion}}
    \label{fig:asus_xtion}
\end{figure}



\subsection{RPLIDAR A2}
\label{subsec:rplidar_a2}
% ** RPLIDAR

Se trata de un láser de 360º con un rango de medida desde 0.2m hasta 16m y una frecuencia de muestreo que se puede ajustar desde 5Hz hasta 15Hz. Usando los drivers mencionados en el apartado del Turtlebot2 (\ref{subsec:turtlebot2_base}) encontraremos un paquete para poder activar y usar este sensor con ROS2.\\

\begin{figure} [H]
    \begin{center}
        \includegraphics[width=5cm]{figs/c3/rplidar-a2.jpg}
    \end{center}
    \caption[RPLIDAR A2]{Sensor RPLIDAR A2. Imagen obtenida de \cite{rplidar}}
    \label{fig:rplidar}
\end{figure}


\section{VisualCircuit}
\label{sec:visualcircuit}
% ** VISUALCIRCUIT

VisualCircuit\footnote{\textbf{VisualCircuit Docs}: \url{https://jderobot.github.io/VisualCircuit/}} es un editor visual online basado en programación por bloques de código orientado al desarrollo de aplicaciones robóticas. Está desarrollado sobre IceStudio\footnote{\textbf{IceStudio Project}: \url{https://github.com/FPGAwars/icestudio}}.
Los bloques que se 




