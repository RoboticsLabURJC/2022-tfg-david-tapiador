\chapter{Herramientas y plataforma de desarrollo}
\label{cap:capitulo3}

\vspace{1cm}

Escribe aquí un párrafo explicando brevemente lo que vas a contar en este capítulo. En este capítulo, explica qué has usado a nivel hardware y software para poder desarrollar tu trabajo: librerías, sistemas operativos, plataformas, entornos de desarrollo, etc.

\section{Lenguaje de programación}
\label{sec:lenguaje_programación}

\begin{code}[H]
    \begin{lstlisting}[language=python]
    print("Hello World")
    \end{lstlisting}
    \caption[Hola mundo en python]{Hola mundo en python}
    \label{cod:holamundo_python}
\end{code}


\section{ROS2 (Robot Operating System 2)}
\label{sec:ros2}

ROS\footnote{\textbf{ROS}: \url{http://wiki.ros.org/es}} o Robot Operating System es un \textit{middleware}\footnote{{\textbf{Middleware}: software que se sitúa entre las aplicaciones y el sistema operativo}} formado por un conjunto de herramientas y librerías de software libre empleadas para el desarrollo de aplicaciones robóticas. Su objetivo es ofrecer una plataforma estándar para todas las ramas de la robótica.\\

ROS se basa en una arquitectura \textit{cliente-servidor} centralizado que, mediante suscriptores y publicadores, permite enviar información, ya sean medidas de sensores, cambios de estado, decisiones usando árboles de decisión, órdenes a los actuadores, etc.\\

Para comunicarse con los servidores (o como se llaman en ROS, "\textit{topics}) se usan nodos. Estos nodos pueden contar con varios publicadores y suscriptores simultáneaos.
Cada \textit{topic} se define con un tipo de mensaje, que será el único que se pueda enviar y recibir a través de él. Estos tipos de mensajes pueden ser mensajes simples como una cadena de caracteres o tipos compuestos con otros tipos, permitiéndonos crear topics adecuados a las necesidades de cada proyecto.\\

\begin{figure} [H]
    \begin{center}
        \includegraphics[width=7cm]{figs/c3/ros_comunicación.png}
    \end{center}
    \caption[Comunicación del nodo Master con los nodos Intermedios y con distintos sensores y actuadores.]{Comunicación del nodo Master con los nodos Intermedios y con distintos sensores y actuadores.. Imagen obtenida de \cite{comunicacion_ros2}}
    \label{fig:ros_master_comunicacion}
\end{figure}




\section{Gazebo}
\label{sec:gazebo}
\section{Turtlebot2}
\label{sec:turtlebot2}
\section{VisualCircuit}
\label{sec:visualcircuit}